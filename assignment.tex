\documentclass{article}
\usepackage{hyperref}

\begin{document}


\title{Erlang introduction}
\author{Arjan Scherpenisse}

\maketitle
\vspace*{2mm}

\section{A simple chat system}

Use {\tt spawn/1} and {\tt register/2} to start a named server process
which maintains a list of connected clients in its state.  Every
message that the server received, needs to be sent to every other
client. Using {\tt register/2} you can give the server process a name,
so each client  connect to it. The client prints every received
message.

Launch different terminals with different erlang shells to test your
code: one terminal for the server, and one for each client. Use {\tt "erl
-sname <nodename>"} to start the erlang shells so the erlang processes
are connected. Use the syntax {\tt \{Processname, NodeName\} ! Message} to
send a message. For instance: {\tt \{server, server@localhost\} ! "foo"} to send {\tt "foo"}
 to the registered process called {\tt server} in the {\tt "erl -sname server"}
terminal.

\section{Possible extensions}

\begin{enumerate}

\item Before sending, echo the message locally using {\tt
  io:format/2}. Then, send the message to every other client {\bf but}
  the sending client, to disable local echo.

\item Require a nickname which clients have to enter when
  registering. Send this nickname along while broadcasting client's
  messages.

\item In the server, maintain a history of the last $N$ messages (with
  timestamps!) that have been sent, and send these upon establishing
  connection.

\item Implement the concept of {\em rooms}: clients can join specific
  rooms and only receive messages sent to those rooms. Find a nice way
  to maintain the list of rooms and clients on the server.

\item For more inspiration: look at the
  \href{http://tools.ietf.org/html/rfc1459#section-3}{features of the
    IRC protocol}, or, even better, the
  \href{http://xmpp.org/extensions/xep-0045.html}{XMPP (Jabber) MUC}
  (Multi User Chat) specification to implement more features in your
  chat system.

\item Make the server/client communicate with eachother over TCP/IP
  like a proper internet server; use the
  \href{http://ftp.csd.uu.se/pub/mirror/erlang/doc/man/gen\_tcp.html}{{\tt
      gen\_tcp}} OTP module. This is a major step: you need to
  implement proper protocol parsing, no more passing around Erlang
  terms between client/server...

\item Implement a web interface for your module, implementing realtime
  web chat using websockets or longpolling. Woah! This sounds more
  complicated than it is. Ask Arjan or Atilla about \href{http://zotonic.com/}{Zotonic}, and/or
  check out the \href{http://code.google.com/p/zchat/}{zchat} module from Google code.

 \item For extra brownie points, document your code, put it in a DVCS,
   publish it on Github, tweet about it, et cetera :-)

\end{enumerate}

\end{document}
