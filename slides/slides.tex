\documentclass{beamer}
 
\mode<presentation>
{
  \usetheme{Berlin}
}

\usepackage[english]{babel}
\usepackage[utf8]{inputenc}

\usepackage{times}
\usepackage[T1]{fontenc}

\title[Erlang: introduction] % (optional, use only with long paper titles)
{An introduction to Erlang}

\author{Arjan Scherpenisse}

\date{July 7, 2011}

\usepackage{pxfonts}
\usepackage{listings}

%% \AtBeginSection[]
%% {
%%   \begin{frame}<beamer>{Outline}
%%     \tableofcontents[currentsection,currentsubsection]
%%   \end{frame}
%% }

\begin{document}

\lstset{language=erlang,
        numbers=left,
        numberstyle=\tiny,
        showstringspaces=false,
        aboveskip=-40pt,
        frame=leftline,
        escapechar=\#
        }


\begin{frame}
  \titlepage
\end{frame}

\begin{frame}{About me}
  \begin{itemize}
  \item Studied AI at the Uva, graduated 2005
  \item Went to Rietveld Academy after
  \item Working at Mediamatic Lab
  \item Soon will start freelancing
  \end{itemize}
\end{frame}

\begin{frame}{Outline}
  \tableofcontents
  % You might wish to add the option [pausesections]
\end{frame}

\begin{frame}{Erlang..?}
  \begin{itemize}
  \item Invented in the 1980's by Joe Armstrong for Ericsson
  \item Open-sourced in 1998
  \item Functional language
  \item Concurrency
  \item Fault-tolerant
  \end{itemize}

\end{frame}


\section{Hello, world}

\begin{frame}[fragile]
\frametitle{How it all starts}
  \begin{semiverbatim}
    \begin{lstlisting}
-module(hello).
-export([say/0]).

say() ->
    io:format("Hello, world!~n").
    \end{lstlisting}
  \end{semiverbatim}
\end{frame}

\begin{frame}[fragile]
Run with:

  \begin{semiverbatim}
# erl
Erlang R13B03 (erts-5.7.4) [source] 

Eshell V5.7.4  (abort with ^G)
1> c(hello).
{ok,hello}
2> hello:say().
Hello, world!
ok
3> 
  \end{semiverbatim}
  
  \begin{itemize}
  \item The interactive shell is handy!  
  \end{itemize}

\end{frame}


\section{Datatypes}

\begin{frame}[fragile]
  \frametitle{Numbers}

  \begin{itemize}
  \item Numbers, you know
  \end{itemize}

  \begin{semiverbatim}
1> 1.
1
2> 1+1.
2
3> 1+1.22.
2.2199999999999998

  \end{semiverbatim}

\end{frame}


\begin{frame}[fragile]
  \frametitle{Atoms}
  \begin{itemize}
  \item Symbolic names, like constants
  \item Starting with lowercase letter
  \item Optionally escaped in {\em single} quotes.
  \end{itemize}
  \begin{semiverbatim}
4> foo.
foo
5> 'foo bar'.
'foo bar'
6> foo+bar.
** exception error: bad argument in an arithmetic expression
     in operator  +/2
        called as foo + bar
7> foo == foo.
true
  \end{semiverbatim}

\end{frame}


\begin{frame}[fragile]
  \frametitle{Booleans}
  \begin{itemize}
  \item Booleans are atoms, too, but allow for boolean logic.
  \end{itemize}

  \begin{semiverbatim}
1> true.
true
2> false.
false
3> false or true.
true
  \end{semiverbatim}

\end{frame}


\begin{frame}[fragile]
  \frametitle{Strings}
  \begin{itemize}
  \item Strings are surrounded by double quotes
  \end{itemize}

  \begin{semiverbatim}
6> "Hello world".
"Hello world"
  \end{semiverbatim}

  \begin{itemize}
  \item In fact, strings are just lists of numbers:
  \end{itemize}

  \begin{semiverbatim}
2> [72,101,108,108,111,32,119,111,114,108,100].
"Hello world"
3> \$H.
72
  \end{semiverbatim}

\end{frame}


\begin{frame}[fragile]
  \frametitle{Lists}
  \begin{itemize}
  \item Variable-length containers
  \item Can contain any other data type
  \item Use {\tt ++} to concatenate 2 lists
  \end{itemize}

  \begin{semiverbatim}
3> [hello, 3.1415, [1,2] ].  
[hello,3.1415,[1,2]]
4> [foo] ++ [bar].
[foo,bar]
  \end{semiverbatim}

\end{frame}


\begin{frame}[fragile]
  \frametitle{Tuples}
  \begin{itemize}
  \item Fixed-length containers
  \item Can contain any other data type
  \item Cannot be changed after declaring
  \end{itemize}

  \begin{semiverbatim}
6> \{1,dd\}.
\{1,dd\}
7> \{1,dd, [a, list, \{tuple\}]\}.
\{1,dd,[a,list,{tuple}]\}
  \end{semiverbatim}

\end{frame}

 
\end{document}
